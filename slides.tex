\documentclass{beamer}

\usetheme[bullet=circle,
          titleline=true,
          pageofpages=of,
          alternativetitlepage=true]{Torino}

\usepackage{color}

\usepackage{ragged2e}
\usepackage{hyphenat}
\usepackage{booktabs}

\usepackage{tikz}

\usetikzlibrary{arrows}
\usetikzlibrary{automata}
\usetikzlibrary{backgrounds}
\usetikzlibrary{decorations}

\usepackage{amsmath}
\usepackage{amsfonts}
\usepackage{amsthm}

\usepackage{highlight/pythonhighlight}

\definecolor{MyGreen}{rgb}{0.40,0.80,0.20}

\title{Understanding importance of automated software testing}
\author{Mateusz Paprocki \texttt{<mattpap@gmail.com>}}
\institute{SymPy Development Team}
\date{\today}

\newenvironment{jblock}[1]{
    \begin{block}{#1}\justifying\nohyphens
}{
    \end{block}
}

\begin{document}

\setbeamercovered{transparent}

\frame{\titlepage}

\begin{frame}[fragile]
    \frametitle{Presentation plan}

    Introduction to software testing:
    \begin{itemize}

        \item Why we need software testing?
            \begin{itemize}
                \item approaches to software testing
            \end{itemize}
        \item What is automated software testing?
        \item Is automation always possible?
        \item What is code coverage?
            \begin{itemize}
                \item common gotchas and pitfalls
            \end{itemize}
    \end{itemize}

    Automated software testing in Python:
    \begin{itemize}
        \item Testing frameworks in Python
        \item Writing tests with py.test
        \item
    \end{itemize}
\end{frame}

\begin{frame}[fragile]
    \frametitle{Verification of computer programs}
    \framesubtitle{Subtitle}

    \begin{itemize}
        \item manual
            \begin{itemize}
                \item very prone to human errors
                \item can be very cheap in some cases (GUIs, web applications)
            \end{itemize}
        \item automated (*)
            \begin{itemize}
                \item write tests once and run many times (different architectures,
                environmental conditions, inputs and outputs (parametrization))
                \item allow randomized test suites
            \end{itemize}
        \item proving
            \begin{itemize}
                \item works only for trivial programs
            \end{itemize}
    \end{itemize}
\end{frame}

\begin{frame}[fragile]
    \frametitle{Humans and software testing}
    \framesubtitle{Subtitle}

    \begin{equation*}
        \mbox{automated} \not= \mbox{autonomous}
    \end{equation*}
    \vskip+0.5cm
    Automated software testing \structure{requires human assist} to:
    \begin{itemize}
        \item create test suites
        \item analyze test results
    \end{itemize}
\end{frame}

\begin{frame}[fragile]
    \frametitle{Approaches to software testing}
    \framesubtitle{Black--Box and White--Box}

    Black--Box testing:
    \begin{itemize}
        \item data-driven testing
        \item unconcerned about internal behavior and structure of the program
        \item find IOs for which program doesn't conform to the specification
        \item {\color{red} cons}: requires exhaustive input testing which is impossible
    \end{itemize}
    White--Box testing:
    \begin{itemize}
        \item logic-driven testing
        \item permits to examine internal structure of the program
        \item derive test data (IOs) from examination of program's logic
        \item {\color{red} cons}: very of then neglects specification of the program
    \end{itemize}
\end{frame}

\begin{frame}[fragile]
    \frametitle{Testing all execution paths}
    \framesubtitle{Subtitle}

    \begin{itemize}
        \item
    \end{itemize}
\end{frame}

\begin{frame}[fragile]
    \frametitle{Program testing guidelines}
    \framesubtitle{Subtitle}

    \begin{itemize}
        \item A necessary part of a test case is a definition of the expected
        output or result.
        \item A programmer should avoid attempting to test his own program.
        \item A programming organization should not test its own programs.
        \item Thoroughly inspect the results of each test.
        \item Test cases must be written for input conditions that are
            \structure{invalid and unexpected}, as well as for those that are
            \structure{valid and expected}.
        \item Examining a program to see if it does not do what it is supposed
        to do is only half the battle; the other half is seeing whether
        the program does what it is not supposed to do.
    \end{itemize}
\end{frame}

\begin{frame}[fragile]
    \frametitle{Program testing guidelines}
    \framesubtitle{Subtitle}

    \begin{itemize}
        \item Avoid throwaway test cases unless the program is truly a
        throwaway program.
        \item Do not plan a testing effort under the tacit assumption that no
        errors will be found.
        \item The probability of the existence of more errors in a section of a
        program is proportional to the number of errors already found in
        that section.
        \item Testing is an extremely creative and intellectually challenging
        task.
    \end{itemize}
\end{frame}

\begin{frame}[fragile]
    \frametitle{Software testing frameworks in Python}
    \framesubtitle{Subtitle}

    \begin{itemize}
        \item unittest
        \item nosetests
        \item py.test
    \end{itemize}
\end{frame}

\begin{frame}[fragile]
    \frametitle{Introduction to py.test}
    \framesubtitle{Subtitle}

    \begin{python}
# content of test_example.py
def func(x):
    return x + 1

def test_func():
    assert func(3) == 4
    \end{python}
\end{frame}

\begin{frame}[fragile]
    \frametitle{Testing exceptions}
    \framesubtitle{Subtitle}

    \begin{python}
# content of test_example.py
import pytest

def func(x):
    if x <= 0:
        raise ValueError("positive value expected")
    else:
        return x + 1

def test_func():
    assert func(3) == 4

    with pytest.raises(ValueError):
        func(-1)
    \end{python}
\end{frame}

\begin{frame}[fragile]
    \frametitle{Literature}
    \framesubtitle{Subtitle}

    Glenford J. Myers
    The art of software testing, 2nd edition,
    John Wiley and Sons, Inc., 2004
\end{frame}

\begin{frame}[fragile]
    \frametitle{Title}
    \framesubtitle{Subtitle}

    \begin{itemize}
        \item item
    \end{itemize}
    \pause
    \begin{python}
    code
    \end{python}
\end{frame}

\begin{frame}
    \begin{center}
        \vskip+0.5cm
        \textbf{\LARGE Thank you for your attention!}
    \end{center}
\end{frame}

\end{document}
