\documentclass{beamer}

\usetheme[bullet=circle,
          titleline=true,
          pageofpages=of,
          alternativetitlepage=true]{Torino}

\usepackage{color}

\usepackage{ragged2e}
\usepackage{hyphenat}
\usepackage{booktabs}

\usepackage{tikz}

\usetikzlibrary{arrows}
\usetikzlibrary{automata}
\usetikzlibrary{backgrounds}
\usetikzlibrary{decorations}

\usepackage{amsmath}
\usepackage{amsfonts}
\usepackage{amsthm}

\usepackage{highlight/pythonhighlight}

\definecolor{MyGreen}{rgb}{0.40,0.80,0.20}

\title{Understanding importance of automated software testing}
\author{Mateusz Paprocki \texttt{<mattpap@gmail.com>}}
\institute{SymPy Development Team}
\date{\today}

\newenvironment{jblock}[1]{
    \begin{block}{#1}\justifying\nohyphens
}{
    \end{block}
}

\begin{document}

\setbeamercovered{transparent}

\frame{\titlepage}

\begin{frame}[fragile]
    \frametitle{Presentation plan}

    \begin{itemize}
        \item Why we need software testing?
        \item Approaches to software testing
        \item Guidelines for proper software testing
        \item Automated software testing in Python:
            \begin{itemize}
                \item Testing frameworks in Python
                \item Writing tests with py.test
                \item Testing web applications with selenium
            \end{itemize}
        \item Analyzing code coverage with tests:
            \begin{itemize}
                \item the \pyth{coverage} module
                \item common gotchas and pitfalls
            \end{itemize}
    \end{itemize}
\end{frame}

\begin{frame}[fragile]
    \frametitle{Why we need software testing?}

    \begin{itemize}
        \item useful software is large and very, very complex
        \item no one knows about every aspect of the project
        \item the same software has to run on different devices,
        platforms, operating systems, etc.
        \item humans tend to make mistakes
        \item your job or even life (if you use the software
        yourself) may depend on software correctness
    \end{itemize}
\end{frame}

\begin{frame}[fragile]
    \frametitle{Verification of computer programs}

    \begin{itemize}
        \item manual
            \begin{itemize}
                \item very prone to human errors
                \item requires detailed software specification
                \item can be very cheap in some cases (testing GUIs, web applications
                and other interactive software)
            \end{itemize}
        \pause
        \item automated (*)
            \begin{itemize}
                \item write tests once and run many times (different architectures,
                operating systems, environmental conditions, etc.)
                \item allow randomized test suites and parametrization of tests
            \end{itemize}
        \pause
        \item formal methods
            \begin{itemize}
                \item doesn't work well for large programs
                \item very expensive and time demanding
                \item used when cost of failure is extremely high
            \end{itemize}
    \end{itemize}
\end{frame}

\begin{frame}[fragile]
    \frametitle{Humans and software testing}

    \begin{equation*}
        \mbox{automated} \not= \mbox{autonomous}
    \end{equation*}
    \vskip+0.5cm
    Automated software testing \structure{requires human assist} to:
    \begin{itemize}
        \item create test suites
        \item analyze test results
    \end{itemize}
\end{frame}

\begin{frame}[fragile]
    \frametitle{Approaches to software testing (1)}

    \structure{Black--Box} testing:
    \begin{itemize}
        \item data-driven testing
        \item unconcerned about internal behavior and structure of the program
        \item find IOs for which program doesn't conform to the specification
        \item {\color{red} cons}: requires exhaustive input testing which is impossible
    \end{itemize}
    \pause
    \structure{White--Box} testing:
    \begin{itemize}
        \item logic-driven testing
        \item permits to examine internal structure of the program
        \item derive test data (IOs) from examination of program's logic
        \item {\color{red} cons}: very of then neglects specification of the program
    \end{itemize}
\end{frame}

\begin{frame}[fragile]
    \frametitle{Approaches to software testing (2)}

    \structure{bottom--up} testing:
    \begin{itemize}
        \item test terminal modules first and then proceed to modules with dependencies
        \item {\color{green} pros}: encourages incremental development of software
        \item {\color{red} cons}: the skeletal program is available very late
    \end{itemize}
    \pause
    \structure{top-down} testing:
    \begin{itemize}
        \item test top--level modules first and then proceed with other modules
        \item {\color{green} pros}: the "big picture" is visible very soon
        \item {\color{red} cons}: no straightforward way to choose order of modules for testing
    \end{itemize}
\end{frame}

\begin{frame}[fragile]
    \frametitle{Program testing guidelines (1)}

    \begin{itemize}
        \item A necessary part of a test case is a definition of the expected
        output or result.
        \pause
        \item A programmer should avoid attempting to test his own program.
        \pause
        \item A programming organization should not test its own programs.
        \pause
        \item Thoroughly inspect the results of each test.
        \pause
        \item Test cases must be written for input conditions that are
            \structure{invalid and unexpected}, as well as for those that are
            \structure{valid and expected}.
        \item Examining a program to see if it does not do what it \structure{is supposed
        to do} is only half the battle; the other half is seeing whether
        the program does what it \structure{is not supposed to do}.
    \end{itemize}
\end{frame}

\begin{frame}[fragile]
    \frametitle{Program testing guidelines (2)}

    \begin{itemize}
        \item Avoid throwaway test cases unless the program is truly a
        throwaway program.
        \pause
        \item Do not plan a testing effort under the tacit assumption that no
        errors will be found.
        \pause
        \item The probability of the existence of more errors in a section of a
        program is proportional to the number of errors already found in
        that section.
        \pause
        \item Testing is an extremely creative and intellectually challenging
        task.
    \end{itemize}
\end{frame}

\begin{frame}[fragile]
    \frametitle{Software testing frameworks in Python}

    \begin{columns}
        \begin{column}[l]{0.3\textwidth}
            \begin{itemize}
                \item unittest
                \item py.test (*)
                \item nose
            \end{itemize}
        \end{column}
        \begin{column}[l]{0.5\textwidth}
            \includegraphics[scale=0.4]{images/testing.jpg}
        \end{column}
    \end{columns}
\end{frame}

\begin{frame}[fragile]
    \frametitle{Introduction to py.test}

    \begin{itemize}
        \item mature, well supported tool
        \item provides no-boilerplate testing
        \item support complex test setups
        \item integrates many common testing methods
    \end{itemize}
    \pause
    Example:
    \begin{python}
def add_one(x):
    return x + 1

def test_add_one():
    assert add_one(3) == 4
    \end{python}
\end{frame}

\begin{frame}[fragile]
    \frametitle{Testing exceptions with py.test}

    \begin{python}
import pytest

def add_one(x):
    if x <= 0:
        raise ValueError("positive value expected")
    else:
        return x + 1

def test_add_one():
    assert add_one(3) == 4

    with pytest.raises(ValueError):
        add_one(-1)
    \end{python}
\end{frame}

\begin{frame}[fragile]
    \frametitle{Testing web applications (1)}

    \begin{itemize}
        \item testing software is not always as simple as executing a function
        or a block of code and compare its result to a well known value or
        \item web applications (getting more and more popular and important
        recently) testing is very demanding:
            \pause
            \begin{itemize}
                \item there are different web browsers
                \item web browsers often disregard web standards
                \item we have to test features like:
                    \pause
                    \begin{itemize}
                        \item clicking/hovering elements
                        \item form submission
                        \item drag \& drop
                        \item AJAX, etc.
                    \end{itemize}
            \end{itemize}
    \end{itemize}
\end{frame}

\begin{frame}[fragile]
    \frametitle{Testing web applications (2)}

    Using \structure{\pyth{selenium}}:
    \begin{python}
from selenium import selenium
sel = selenium('localhost', 4444, '*firefox', 'url')

sel.start()
sel.open('page')

sel.type("css=input[name=q]", "abc")
sel.submit("css=form.search")

sel.is_text_present("from sympy.abc import x")

sel.stop()
    \end{python}
\end{frame}

\begin{frame}[fragile]
    \frametitle{Testing web applications (3)}

In a script we have to give a page chance to load:
    \begin{python}
from selenium import selenium
sel = selenium('localhost', 4444, '*firefox', 'url')
sel.start()
sel.open('page')
sel.wait_for_page_to_load(10000) # milliseconds
...
    \end{python}
\pause
We can also turn this code into a test case:
    \begin{python}
...
sel.type("css=input[name=q]", "abc")
sel.submit("css=form.search")
sel.wait_for_page_to_load(10000) # milliseconds
assert sel.is_text_present("from sympy.abc import x")
...
    \end{python}
\end{frame}

\begin{frame}[fragile]
    \frametitle{Code coverage with tests (1)}

    Lets define a simple function for \structure{finding minimum
    and maximum values} of iterable containers \structure{simultaneously}:
    \begin{python}
def minmax(*args):
    """Find minimum and maximum values simultaneously. """
    if len(args) > 1:
        return minmax(args)
    iterable = iter(args[0])
    min = max = iterable.next()
    for item in iterable:
        if item < min:
            min = item
        elif item > max:
            max = item
    return min, max
    \end{python}
\end{frame}

\begin{frame}[fragile]
    \frametitle{Code coverage with tests (2)}

    We can \structure{experiment} with the function in the terminal:
    \begin{python}
 >>> minmax(2, 1, 4, 3)
(1, 4)
 >>> minmax([2, 1, 4, 3])
(1, 4)
    \end{python}
    \pause
    We can also \structure{write a test suite} for \pyth{minmax()}:
    \begin{python}
def test_minmax():
    assert minmax(1, 2, 3, 4) == (1, 4)
    assert minmax([1, 2, 3, 4]) == (1, 4)
    \end{python}
    \pause
    For future reference, I put \pyth{minmax()} and \pyth{test_minmax()}
    in \verb+test_minmax.py+.
\end{frame}

\begin{frame}[fragile]
    \frametitle{Code coverage with tests (3)}

    Lets use \structure{\pyth{coverage}} module to see how many lines of
    \verb+test_minmax.py+ are covered after executing \pyth{test_minmax()}:
    \begin{python}
import coverage
cov = coverage.coverage(source=["test_minmax.py"])
cov.start()
import test_minmax
test_minmax.test_minmax()
cov.stop()
print cov.report()
Name          Stmts   Miss  Cover   Missing
-------------------------------------------
test_minmax      14      0   100%
    \end{python}
    \pause
    Nice, \structure{100\% code coverage}.
\end{frame}

\begin{frame}[fragile]
    \frametitle{Code coverage with tests (4)}

    However, 100\% code coverage with tests doesn't imply in this case
    that \pyth{minmax()} handle corner cases well:
    \begin{python}
In [1]: from test_minmax import minmax

In [2]: minmax()
...
IndexError: tuple index out of range

In [3]: minmax([])
...
StopIteration:
    \end{python}
    \pause
    This doesn't look appealing to the end user and \structure{exposes too much
    unnecessary details about the internal implementation} of \pyth{minmax()}.
\end{frame}

\begin{frame}[fragile]
    \frametitle{Code coverage with tests (5)}

    100\% code coverage with tests:
    \begin{itemize}
        \item doesn't imply correctness of code or conformance to specification
        \pause
        \item is very important in interpreted programming languages:
            \begin{itemize}
                \item there's high chance that trivial mistakes (e.g. typos) exist
                in untested branches and will crash the program with \pyth{NameError},
                \pyth{ImportError} or other exception
                \item using audit tools like \verb+pyflakes+ and editor bindings
                for them (e.g. \verb+vim-pyflakes+) can help a lot, but doesn't
                imply that 100\% code coverage is unimportant
            \end{itemize}
    \end{itemize}
\end{frame}

\begin{frame}[fragile]
    \frametitle{Code coverage with tests (6)}

    {\color{green} do}:
    \begin{itemize}
        \item make your goal to cover all branches of code to avoid
        trivial but expensive mistakes and bugs
        \item use bottom--up approach to make sure that your tests
        verify not only the "big picture" but also all functions
        and modules alone
        \item code coverage of a particular unit should approach
        100\% no matter whether you test the whole project or just
        the unit alone
    \end{itemize}
    \pause
    {\color{red} don't}:
    \begin{itemize}
        \item assume that 100\% code coverage means correct programs
        \item enforce 100\% coverage by trivializing the test suite
    \end{itemize}
\end{frame}

\begin{frame}[fragile]
    \frametitle{Code coverage with tests (7)}

    100\% code coverage with tests won't help you to prevent bugs:
    \begin{itemize}
        \item that are caused by missing branches in the code
        \pause
        \item that show up for certain inputs, e.g.:
            \begin{itemize}
                \item very large or very small numbers (overflow, underflow)
                \item large data structures (memory overrun, recursion depth)
            \end{itemize}
        \pause
        \item that show up on certain platforms, e.g.:
            \begin{itemize}
                \item missing modules, API functions (e.g. posix on Windows)
            \end{itemize}
    \end{itemize}
\end{frame}

\begin{frame}[fragile]
    \frametitle{Testing by example (1)}

    Lets revise \pyth{minmax()} and improve its docstring:
    \begin{python}
def minmax(*args):
    """
    Find minimum and maximum values simultaneously.

    Example
    -------
    >>> minmax(2, 1, 4, 3)
    (1, 4)
    >>> minmax([2, 1, 4, 3])
    (1, 4)

    """
    ...
    \end{python}
\end{frame}

\begin{frame}[fragile]
    \frametitle{Testing by example (2)}

    We can verify correctness of examples using \structure{\pyth{doctest}} module:
    \begin{python}
 >>> import doctest
 >>> import test_minmax
 >>> doctest.testmod(test_minmax)
TestResults(failed=0, attempted=2)
    \end{python}
    \pause
    This tells us that all doctests in \verb+test_minmax.py+ are correct.
\end{frame}

\begin{frame}[fragile]
    \frametitle{Testing by example (3)}

    If we make a mistake in an example, e.g.:
    \begin{python}
def minmax(*args):
    """
    Find minimum and maximum values simultaneously.

    Example
    -------
    >>> minmax(2, 1, 4, 3)
    (1, 4)
    >>> minmax([2, 1, 4, 3])
    (3, 4)

    """
    ...
    \end{python}
\end{frame}

\begin{frame}[fragile]
    \frametitle{Testing by example (4)}

    \pyth{doctest} will show us what is the problem:
    \begin{python}
 >>> import doctest, test_minmax
 >>> doctest.testmod(test_minmax)
********************************************************
File "test_minmax.py", line 9, in test_minmax.minmax
Failed example:
    minmax([2, 1, 4, 3])
Expected:
    (3, 4)
Got:
    (1, 4)
********************************************************
1 items had failures:
   1 of   2 in test_minmax.minmax
***Test Failed*** 1 failures.
TestResults(failed=1, attempted=2)
    \end{python}
\end{frame}

\begin{frame}[fragile]
    \frametitle{Testing by example (5)}

    {\color{green} pros:}
    \begin{itemize}
        \item allows for automatic verification that examples work
        \item gives limited code testing ability
        \item adds quality to documentation
    \end{itemize}
    \pause
    {\color{red} cons:}
    \begin{itemize}
        \item relies on comparing texts, so is very fragile in nontrivial cases
        \item used improperly can make docstrings long and unpleasant to read
    \end{itemize}
\end{frame}

\begin{frame}[fragile]
    \frametitle{Summing up}

    \begin{itemize}
        \item testing is an important part of software development process
        \pause
        \item better testing means better software means less work in the
        long term and more satisfied customers
        \pause
        \item Python has excellent tools for automation of testing process
        \pause
        \item there are tools for advanced automated testing setups (GUIs,
        web applications, web servers, etc.)
    \end{itemize}
\end{frame}

\begin{frame}[fragile]
    \frametitle{Literature}

    \begin{itemize}
        \item Glenford J. Myers, \emph{The art of software testing}, 2nd edition, John Wiley and Sons, Inc., 2004
        \item http://docs.python.org/library/doctest.html
        \item http://www.seleniumhq.org
        \item http://pytest.org
    \end{itemize}
\end{frame}

\begin{frame}
    \begin{center}
        \vskip+0.5cm
        \textbf{\LARGE Thank you for your attention!}
    \end{center}
\end{frame}

%\begin{frame}[fragile]
%    \frametitle{Title}
%
%    \begin{itemize}
%        \item item
%    \end{itemize}
%    \pause
%    \begin{python}
%    code
%    \end{python}
%\end{frame}

\end{document}
